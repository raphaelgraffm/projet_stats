\documentclass[a4paper,12pt]{article}

 \usepackage[francais]{babel}
 \usepackage[T1]{fontenc}
 \usepackage[utf8]{inputenc}
 \usepackage[top=2cm,bottom=2cm,left=2cm,right=2cm]{geometry}
 \usepackage[table,xcdraw]{xcolor}
 \usepackage{graphicx}
 \usepackage{amsmath}
 \usepackage{amssymb}
 \usepackage{listings}
 \usepackage{eurosym}

 \lstset{language=Python}

 \title{Statistiques numériques et analyse de données - Projet}
 \author{Alexandre AHETO, Raphaël GRAFF-MENTZINGER, Anne SPITZ}
 \date{5 février 2017}

 \begin{document}

 \maketitle

Dans le cadre de ce projet, nous avons choisi d'étudier la base de données \og Communities and Crime Data Set \fg{}, fournie librement par le site UCI. Composée de 1994 échantillons comportant chacun 128 attributs réels, elle rassemble des informations diverses, notamment démographiques, sur les \emph{communities} américaines. On y retrouve ainsi des données e dernier attribut de chaque échantillon représente son taux de criminalité 

 \end{document}
